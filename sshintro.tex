\documentclass{article}

\usepackage[utf8]{inputenc}
\usepackage[bulgarian]{babel}

\title{Въведение в SSH}
\author{rambius}

\begin{document}
\maketitle
\begin{abstract}
Целта на този документ е да опише възможностите на програмата OpenSSH и на протокола Secure Shell, който тя реализира.
\end{abstract}

\section{Какво са SSH и OpenSSH}
SSH (Secure SHell) е мрежов протокол, който предава данните си в криптиран вид. Най-често се използва за изпълнение на команди и за прехвърляне на файлове на отдалечена машина. Криптирането на данните между двете крайни точки ги предпазва от подслушване на комуникацията.

Най-простият пример за употреба е следният. Почти всеки Unix администратор трябва да се включва към сървърите си от работната си машина, за да ги поддържа. Той/тя отваря SSH сесия към сървъра, автенцира се, и изпълнява различни команди. Автентикацията може да се извърши по няколко начина, включително и с име и парола\footnote{Паролите смърдят. Една от причините да напишем този документ е да покажем как да не се използват пароли.}. SSH предпазва данните със самоличността на потребителя, като ги криптира. По-стари подобни протоколи като telnet, rlogin и други изпращат тази информация в явен текст, което я прави уязвима за подслушване и кражба. Има случаи, когато данните не са толкова тайни - например чрез SSH могат да се споделят файлове между различни потребители на различни машини. Щом ги изпращаме е ясно, че не ги пазим в тайна, но и в този случай искаме автентикацията да бъде сигурна и защитена дори и от получателя на файла.

OpenSSH е софтуер, който реализира SSH протокола. Това е една от най-популярните имплементации разработена от проекта OpenBSD и е включен във всички BSD операционни системи и почти всички Linux дистрибуции.

\section{Първи стъпки}
Ще покажем най-простата употреба на SSH - как да се включим към машина с потребителско име и парола. За улеснение ще използва localhost като отдалечена машина, въпреки че тя не е.

SSH има стандартна клиент - сървър архитектура. На машината, към която искаме да се включим, трябва да има стартиран SSH сървър или демон. Името му при OpenSSH е sshd. Проверяваме с командата pgrep дали има такъв процес:
\begin{verbatim}
$ pgrep -lf sshd
45589 /usr/sbin/sshd
\end{verbatim}
Изходът показва, че sshd демонът е стартиран. Ако няма такъв процес, то най-вероятно sshd трябва да бъде стартиран. Проверете man страницата на sshd за подробности как да стане това.  След това използваме командата ssh, за да се свържем:
\begin{verbatim}
\end{verbatim}
\end{document}